% Created 2016-10-11 Tue 14:21
\documentclass[12pt]{article}
\usepackage[utf8]{inputenc}
\usepackage[T1]{fontenc}
\usepackage{fixltx2e}
\usepackage{graphicx}
\usepackage{longtable}
\usepackage{float}
\usepackage{wrapfig}
\usepackage{rotating}
\usepackage[normalem]{ulem}
\usepackage{amsmath}
\usepackage{textcomp}
\usepackage{marvosym}
\usepackage{wasysym}
\usepackage{amssymb}
\usepackage{hyperref}
\tolerance=1000
\usepackage{listings}
\usepackage{geometry,listings,amsmath,amssymb,amsthm}
\author{Daniel Dyla}
\date{\today}
\title{CSE 361 HW-3}
\hypersetup{
  pdfkeywords={},
  pdfsubject={},
  pdfcreator={Emacs 24.5.1 (Org mode 8.2.10)}}
\begin{document}

\maketitle

\section{Problem Statement}
\label{sec-1}

Given a list of integers, return the 3 elements that produce the
largest product.

\section{Code}
\label{sec-2}

\lstset{language=Python,label= ,caption= ,numbers=none}
\begin{lstlisting}
def read_dat(fname):
    with open(fname) as f:
	for l in f.readlines():
	    yield([int(s.strip()) for s in l.split(',')])

def max_3prod(a):
    n = len(a)
    if n < 3:
	return
    ml = a[:3]
    m = a[0] * a[1] * a[2]
    for i in range(n-2):
	for j in range(i+1, n-1):
	    for k in range(j+1, n):
		t = m
		m = max(m, a[i] * a[j] * a[k])
		if m > t:
		    ml = [a[i], a[j], a[k]]
    return ml, m


s = 0
for i in read_dat('sum-of-k.data'):
    lst, p = max_3prod(i)
    s += p
    print '[' + " ".join(str(i) for i in lst) + ']  ->', p

print "TOTAL:", s
\end{lstlisting}

\section{Output}
\label{sec-3}

\begin{verbatim}
[1 2 3]  -> 6
[-1 -2 -3]  -> -6
[2 3 4]  -> 24
[2 3 4]  -> 24
[1 3 4]  -> 12
[1 2 4]  -> 8
[1 2 3]  -> 6
[-1 -2 4]  -> 8
[-1 -3 4]  -> 12
[-1 3 -4]  -> 12
[-2 -3 4]  -> 24
[-2 3 -4]  -> 24
[2 -3 -4]  -> 24
[-1 -2 -3]  -> -6
[1 -3 -4]  -> 12
[2 -3 -4]  -> 24
[-2 3 -4]  -> 24
[-2 -3 4]  -> 24
[3 4 5]  -> 60
[-1 -2 -3]  -> -6
[2 -4 -5]  -> 40
[3 4 5]  -> 60
[-2 -3 5]  -> 30
[-2 -4 5]  -> 40
[-3 4 -5]  -> 60
[2 3 4]  -> 24
[3 4 5]  -> 60
[-3 -4 5]  -> 60
[1 -4 -5]  -> 20
[5 6 7]  -> 210
[2 3 4]  -> 24
[2 3 4]  -> 24
[-2 -3 4]  -> 24
[2 -3 -4]  -> 24
[-1 -2 2]  -> 4
[-1 -2 1]  -> 2
[-1 -2 0]  -> 0
[-2 3 -3]  -> 18
[2 -4 -3]  -> 24
[0 -1 -2]  -> 0
[1 -2 -3]  -> 6
[-1 -2 3]  -> 6
TOTAL: 1070
\end{verbatim}

\section{Correctness}
\label{sec-4}

The algorithm terminates because each of the 3 for loops have
pre-defined bounds.

The algorithm begins by assuming the 3 left-most elements of the array
are the maximum product, and stores the product of those 3
elements. It then loops through every possible combination of 3
elements of the array, storing the combination and its product if the
product of the combination is larger than the previous largest
product. Every possible combination is checked, and each is checked
against the previous maximum, so when the algorithm terminates the
largest combination of 3 elements is returned. Thus the algorithm is
correct.

\section{Asymptotic Runtime}
\label{sec-5}

Lines 16-18 (inside the innermost for-loop) perform a constant amount
of work (in the worst case) on each iteration, so it can be counted as
a single operation.

\begin{align}
  &= \sum_{i=0}^{n-3}\sum_{j=i+1}^{n-2}\sum_{k=j+1}^{n-1} 1 \\
  &= \sum_{i=0}^{n-3}\sum_{j=i+1}^{n-2}((n-1) - (j+1) + 1) \\
  &= \sum_{i=0}^{n-3}\sum_{j=i+1}^{n-2}(n - 1 - j) \\
  &= \sum_{i=0}^{n-3}(\sum_{j=i+1}^{n-2}(n - 1) - \sum_{j=i+1}^{n-2} j) \\
  &= \sum_{i=0}^{n-3}((n-1)((n-2)-(i+1)+1) - \sum_{j=i+1}^{n-2} j) \\
  &= \sum_{i=0}^{n-3}((n-1)(n-2-i) - \sum_{j=i+1}^{n-2} j) \\
  &= \sum_{i=0}^{n-3}((n^2 - 2n - in - n + 2 + i) - \sum_{j=i+1}^{n-2} j) \\
  &= \sum_{i=0}^{n-3}((n^2 - 2n - in - n + 2 + i) - (\sum_{j=0}^{n-2} j - \sum_{j=0}^{i}j)) \\
  &= \sum_{i=0}^{n-3}((n^2 - 2n - in - n + 2 + i) - (\sum_{j=0}^{n-2} j - \frac{i(i+1)}{2})) \\
  &= \sum_{i=0}^{n-3}((n^2 - 2n - in - n + 2 + i) - (\frac{(n-2)(n-1)}{2} - \frac{i(i+1)}{2})) \\
  &= \sum_{i=0}^{n-3}((n^2 - 2n - in - n + 2 + i) - (\frac{n^2 - 3n + 2}{2} - \frac{i^2+i}{2})) \\
  &= \sum_{i=0}^{n-3}((n^2 - 2n - in - n + 2 + i) - \frac{n^2 - 3n + 2 - i^2 - i}{2}) \\
  &= \sum_{i=0}^{n-3}(n^2 - 2n - in - n + 2 + i - n^2/2 + 3n/2 - 1 + i^2/2 + i/2) \\  
  &= \frac{1}{6}(n-2)(n-1)n \\
  &\in \Theta(n^3)
\end{align}
% Emacs 24.5.1 (Org mode 8.2.10)
\end{document}
