% Created 2016-09-29 Thu 12:39
\documentclass[12pt]{article}
\usepackage[utf8]{inputenc}
\usepackage[T1]{fontenc}
\usepackage{fixltx2e}
\usepackage{graphicx}
\usepackage{longtable}
\usepackage{float}
\usepackage{wrapfig}
\usepackage{rotating}
\usepackage[normalem]{ulem}
\usepackage{amsmath}
\usepackage{textcomp}
\usepackage{marvosym}
\usepackage{wasysym}
\usepackage{amssymb}
\usepackage{hyperref}
\tolerance=1000
\usepackage{listings}
\usepackage{geometry,listings,amsmath,amssymb,amsthm}
\author{Daniel Dyla}
\date{\today}
\title{CSE 361 HW-2}
\hypersetup{
  pdfkeywords={},
  pdfsubject={},
  pdfcreator={Emacs 24.5.1 (Org mode 8.2.10)}}
\begin{document}

\maketitle

\section{3-2 Relative asymptotic growths}
\label{sec-1}

Indicate, for each pair of expressions $(A,B)$ in the table below,
whether A is O, $\Omega$, or $\Theta$ of B. Assume $k \leq 1,
\epsilon > 0, c > 1$ are constants. Your answer should be in the form
of the table with "yes" or "no" written in each box.

\begin{center}
\begin{tabular}{ll|lll}
A & B & O & $\Omega$ & $\Theta$\\
\hline
$\lg$$^{\text{k}}$(n) & n$^{\epsilon}$ & yes & no & no\\
n$^{\text{k}}$ & c$^{\text{n}}$ & yes & no & no\\
\sqrt{n} & n$^{\text{sin(n)}}$ & no & no & no\\
2$^{\text{n}}$ & 2$^{\text{n/2}}$ & no & yes & no\\
n$^{\lg\text{(c)}}$ & c$^{\lg\text{(n)}}$ & yes & yes & yes\\
$\lg$(n!) & $\lg$(n$^{\text{n}}$) & yes & yes & yes\\
 &  &  &  & \\
\end{tabular}
\end{center}

\section{3-3 Ordering by asymptotic growth rates}
\label{sec-2}

\begin{itemize}
\item Rank the following functions by order of growth; that is, find an
arrangement $g_1, g_2,...,g_{30}$ of the functions satisfying $g_1 \in
   \Omega(g_2), g_2 \in \Omega(g_3),...,g_{29} \in \Omega(g({30})$. Partition
your list into equivalence classes such that the functions $f(n)$
and $g(n)$ are in the same class if and only if $f(n) \in
   \Theta(g(n))$.
\end{itemize}


1

n$^{\text{1/lg n}}$

lg lg$^{\text{*}}$ n

lg$^{\text{*}}$ lg n

lg$^{\text{*}}$ n

2$^{\text{lg n}}$

ln ln n

\sqrt{lg n}

ln n

lg$^{\text{2}}$ n

2$^{\sqrt{2 lg n}}$

\sqrt{n}

\sqrt{2}$^{\text{lg n}}$

2$^{\text{lg n}}$

n

lg(n!)

n lg n

4$^{\text{lg n}}$

n$^{\text{2}}$

n$^{\text{3}}$

(lg n)!

n$^{\text{lg lg n}}$

(lg n)$^{\text{lg n}}$

(3/2)$^{\text{n}}$

2$^{\text{n}}$

n 2$^{\text{n}}$

e$^{\text{n}}$

n!

(n+1)!

2$^{\text{2}^{\text{n}}}$

2$^{\text{2}^{\text{n+1}}}$

\begin{itemize}
\item Give an example of a single non-negative function f(n) such that for all
functions g$_{\text{i}}$(n) in part 1, f(n) is neither O(g(n)) nor
$\Omega$(g(n))
\end{itemize}

n!!$^{\text{sin(n)+1/2}}$

\section{3-4 Asymptotic Notation Properties}
\label{sec-3}

Prove or disprove each of the following conjectures:

\subsection{f(n) $\in$ O(g(n)) implies g(n) $\in$ O(f(n))}
\label{sec-3-1}

False. n $\in$ O(lg(n)) but lg(n) not $\in$ O(n)

\subsection{f(n) + g(n) $\in$ $\Theta$(min(f(n),g(n)))}
\label{sec-3-2}

False. n + n$^{\text{2}}$ not $\in$ $\Theta$(n)

\subsection{f(n) $\in$ O(g(n)) implies lg(f(n)) $\in$ O(lg(g(n)))}
\label{sec-3-3}

???

\subsection{f(n) $\in$ O(g(n)) implies 2$^{\text{f(n)}}$ $\in$ O(2$^{\text{g(n)}}$)}
\label{sec-3-4}

False. f(n) = 2n, g(n) = n. 2$^{\text{2n}}$ not $\in$ O(2$^{\text{n}}$)

\subsection{f(n) $\in$ O((f(n))$^{\text{2}}$)}
\label{sec-3-5}

False. f(n) = 1/n. 1/n not $\in$ O(1/n$^{\text{2}}$)

\subsection{f(n) $\in$ O(g(n)) implies g(n) $\in$ $\Omega$(f(n))}
\label{sec-3-6}

True by definition. f(n) <= C g(n) for positive C, therefore 1/c f(n)
<= g(n)

\subsection{f(n) $\in$ $\Theta$(f(n/2))}
\label{sec-3-7}

2$^{\text{n}}$ not <= C 2$^{\text{n/2}}$ for big C

\subsection{f(n) + \textit{o}(f(n)) $\in$ $\Theta$(f(n))}
\label{sec-3-8}

???
% Emacs 24.5.1 (Org mode 8.2.10)
\end{document}
