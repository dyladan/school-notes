% Created 2016-10-11 Tue 16:56
\documentclass[12pt]{article}
\usepackage[utf8]{inputenc}
\usepackage[T1]{fontenc}
\usepackage{fixltx2e}
\usepackage{graphicx}
\usepackage{longtable}
\usepackage{float}
\usepackage{wrapfig}
\usepackage{rotating}
\usepackage[normalem]{ulem}
\usepackage{amsmath}
\usepackage{textcomp}
\usepackage{marvosym}
\usepackage{wasysym}
\usepackage{amssymb}
\usepackage{hyperref}
\tolerance=1000
\usepackage{listings}
\usepackage{geometry,listings,amsmath,amssymb,amsthm}
\author{Daniel Dyla}
\date{\today}
\title{CSE 361 Lecture Notes}
\hypersetup{
  pdfkeywords={},
  pdfsubject={},
  pdfcreator={Emacs 24.5.1 (Org mode 8.2.10)}}
\begin{document}

\maketitle

\section{Sums}
\label{sec-1}

\subsection{Sum of constant}
\label{sec-1-1}

\begin{align}
\sum_{i=k}^n c &= c + c + c + c + ... + c \\
&= n - k + 1
\end{align}

\subsection{Sum of integers}
\label{sec-1-2}

\begin{align}
\sum_{i=0}^n i &= 0 + 1 + 2 + 3 + ... + n \\
&= \frac{n(n+1)}{2}
\end{align}

\subsection{Sum of squares}
\label{sec-1-3}

\begin{align}
\sum_{i=0}^n i^2 &= 0 + 1 + 4 + 9 + ... + n^2 \\
&= \frac{n(n+1)(2n+1)}{3}
\end{align}


\section{Complexity}
\label{sec-2}

\begin{itemize}
\item Big Theta has upper and lower bound
\item Big Omega has lower bound
\item Big Oh (Omnicron) has upper bound
\end{itemize}

\section{Recursive Complexity}
\label{sec-3}

\begin{enumerate}
\item Take the base case
\item Take other cases
\item Expand and solve
\end{enumerate}

\subsection{Simple Case}
\label{sec-3-1}

How many times is the function $f()$ run?

\lstset{language=Python,label= ,caption= ,numbers=none}
\begin{lstlisting}
def r(n):
  f()
  if n == 1:
    return 5
  else:
    return r(n-1)
\end{lstlisting}

\[ T(n) = 
\begin{cases}
1 & n = 1 \\
1 + T(n-1) & else \\
\end{cases}
\]

\begin{align}
  T(n) &= 1 + T(n-1) \\
  &= 1 + 1 + T(n-2) \\
  &= 1 + 1 + 1 + T(n-3) \\
  &= 1 + 1 + 1 + 1 + T(n-4) \\
  &= 1 + 1 + 1 + 1 + ... + 1 \\
  &= n
\end{align}

\subsection{More Complex Case}
\label{sec-3-2}

How many times is the function $f()$ run?

\lstset{language=Python,label= ,caption= ,numbers=none}
\begin{lstlisting}
def r(n):
  f()
  if n == 1:
    return 5
  else:
    return r(n/2)
\end{lstlisting}

\[ T(n) = 
\begin{cases}
1 & n = 1 \\
1 + T(n/2) & else \\
\end{cases}
\]

\begin{align}
  T(n) &= 1 + T(n/2) \\
  &= 1 + (1 + T(n/4)) \\
  &= 1 + (1 + (1 + T(n/8))) \\
  &= 1 + \log_2 n
\end{align}

\section{prove 2n$^{\text{3}}$ - n$^{\text{2}}$ + n -2 $\in$ $\Theta$(n$^{\text{3}}$)}
\label{sec-4}

Claim: there is a C$_{\text{1}}$, C$_{\text{2}}$, n$_{\text{0}}$ such that $C_1 n^3 <= 2n^3 - n^2 + n
-2<= C_2 n^3 \forall n >= n_0$

C$_{\text{1}}$ = 1, C$_{\text{2}}$ = 3, n$_{\text{0}}$ = 2

\section{Analyze the function}
\label{sec-5}

\lstset{language=Python,label= ,caption= ,numbers=none}
\begin{lstlisting}
def mystery(n):
    t = 0
    for i in range(1, n+1):
	s = 1
	for j in range(1, n+1):
	    s = s * j
	t = t + s
    return t

mystery(5)
\end{lstlisting}

\subsection{What does it do?}
\label{sec-5-1}

The sum of all factorials from 0 to n.

\begin{itemize}
\item The inner loop computes factorials
\item The outer loop sums those factorials
\end{itemize}

\subsection{Runtime}
\label{sec-5-2}

\begin{align}
  T(n) &= 1 + \sum_{i=1}^{n}(2 + \sum_{j=1}^{n} 1) \\
  &= 1 + \sum_{i=1}^{n} (2+n) \\
  &= 1 + n(2+n) \\
  &= n^2 + 2n + 1 \\
  &\in \Theta(n^2)
\end{align}

\section{Lecture 5}
\label{sec-6}

Mostly we talked about HW-1 today. See hw1.org.

\subsection{Problem}
\label{sec-6-1}

Given an increasing function $f(x)$, defined for non-negative x, and a
number $T$, find a number $z \in [0,\infinity]$, such that $f(z) =
T$. $f$ is expensive.

\lstset{language=Python,label= ,caption= ,numbers=none}
\begin{lstlisting}
def increase(n):
    return 2 * n

def binsearch(f, T, a, b, e=0.1):
    while 1:
	m = (float(a) + b) / 2
	fm = f(m)
	err = (fm - T) ** 2
	if (err < e ** 2):
	    return fm
	elif fm < T:
	    a = m
	else:
	    b = m

def find_z(f, T, e=0.1):
    a, b = 0, 1
    while (f(b) < T):
	a = b
	b = increase(b)

    z = binsearch(f, T, a, b)
    return z
\end{lstlisting}

Initial search runs in $f'(T)$. Binary search runs in
$log_2((b-a)/\epsilon)$.

\begin{align}
  R(f, T) &= \log_2(f'(T)) + \log_2(\frac{b-a}{\epsilon}) \\
  &= \log_2(f'(T)) + \log_2(\frac{\frac{f'(T)}{2}}{\epsilon}) \\
  &\in \Theta(\log_2(f'(T)))
\end{align}

\section{Lecture 6}
\label{sec-7}

Given an array of real numbers, find a contiguous subarray with the
largest possible sum.

\lstset{language=Python,label= ,caption= ,numbers=none}
\begin{lstlisting}
def A0(l):
    n = len(l)
    large = l[0]
    for i in range(n):
	for j in range(i,n):
	    large = max(large, sum(l[i:j+1]))
    return large

l = [1, 3, 4, 2, -7, 5]
A0(l)
\end{lstlisting}

\lstset{language=Python,label= ,caption= ,numbers=none}
\begin{lstlisting}
def A1(l):
    n,largest = len(l),0
    for i in range(n):
	s = 0
	for j in range(i,n):
	    s += l[j]
	    largest = max(s,largest)
    return largest

A1(l)
\end{lstlisting}

\lstset{language=Python,label= ,caption= ,numbers=none}
\begin{lstlisting}
def A2(l):
    c = [0]
    for i in range(len(l)):
	c.append(c[i] + l[i])
    largest = 0
    for i in range(len(l)):
	for j in range(i, len(l)):
	    s = c[j+1] - c[i]
	    largest = max(s, largest)
    return largest



A2(l)
\end{lstlisting}

\lstset{language=Python,label= ,caption= ,numbers=none}
\begin{lstlisting}
def A3(a):
    n = len(a)
    m = n//2
    a1 = a[:n]
    a2 = a[n:]
    l = n-1
    r = n
    c = [0]
    for i in range(n):
	c.append(c[i]+a[i])
\end{lstlisting}

\lstset{language=Python,label= ,caption= ,numbers=none}
\begin{lstlisting}
def A4(a):
    mf, mh = 0, 0
    for i in a:
	mh = max(mh+i, 0)
	mf = max(mf,mh)
    return mf

A4(l)
\end{lstlisting}

\subsection{Homework}
\label{sec-7-1}

maximum product of 3 elements in the array

\section{Lecture 7}
\label{sec-8}

\section{Test 1 Prep}
\label{sec-9}

\begin{itemize}
\item Verify Strassen at least once before the test
\item if T(m) >= T'(m), then T(m) $\in$ $\Omega$(T'(m))
\item if T(m) <= T'(m), then T(m) $\in$ O(T'(m))
\end{itemize}

\section{Exam}
\label{sec-10}

\subsection{Page 2}
\label{sec-10-1}

Count the arithmetic operations

sum from 0 to n-1

Don't forget T(n) where n is the length of the array

\subsection{Page 3}
\label{sec-10-2}

It evaluates the polynomial at x

Code is efficient because it is in $\Theta$(n) where the natural way to
evaluate polynomials is in $\Theta$(n$^{\text{2}}$)

Horner's algorithm

9 = C(10$^{\text{4}}$)$^{\text{7/2}}$; 9 = C10$^{\text{14}}$; C = 9x10$^{\text{-14}}$; x = C10$^{\text{14}}$
Solve for x

\subsection{Page 4}
\label{sec-10-3}

Efficient algorithms for the maximum subarray problem by distance 

"kadane's algorithm"

\lstset{language=Python,label= ,caption= ,numbers=none}
\begin{lstlisting}
m = 0
subarray = [[0]]
for each row r1 in matrix:
    for each element e1 in r1:
	for each row r2 below r1:
	    for each element e2 right of e1:
		s = 0
		for each row r3 from r2 to the end:
		    for each element e3 from e2 to the end:
			s += a[r3,e3]
		if s > m:
		    m = s
		    subarray = submatrix(r1,r2,e1,e2)
\end{lstlisting}

\lstset{language=Python,label= ,caption= ,numbers=none}
\begin{lstlisting}
def msum(a):
    m = len(a) # row
    n = len(a[0]) # col
    best = a[0][0]
    idxs = [0,0,0,0]
    for tlrow in range(m):
	for tlcol in range(n):
	    for brrow in range(tlrow,m):
		for brcol in range(tlcol,n):
		    s=arrsum(a,tlrow,tlcol,brrow,brcol)
		    if s > best:
			best,idxs = s, [tlrow,tlcol,brrow,brocl]
    return best

def arrsum(a,tlr,tlc,brr,brc):
    s = 0
    for i in range(tlr,brr+1):
	for j in range(trc,brc+1):
	    s += a[i][j]
    return s

arrsum([[1,2][3,4]],0,0,1,1)
\end{lstlisting}

Count \# of times "s += a[i][j] is called

\subsection{Page 5}
\label{sec-10-4}

All true

set \{g(n) | $\exists$ c, n$_{\text{0}}$>0 g(n) $\le$ cf(n), $\forall$ n $\ge$ n$_{\text{0}}$\}

set \{g(n) | $\exists$ c, n$_{\text{0}}$>0 g(n) $\ge$ cf(n), $\forall$ n $\ge$ n$_{\text{0}}$\}

\[ T(n) = 
\begin{cases}
  1 & n \leq 1 \\
  1 + T(n-1) + T(n-2) + 1 & else \\
\end{cases}
\]
% Emacs 24.5.1 (Org mode 8.2.10)
\end{document}
