% Created 2016-09-18 Sun 21:37
\documentclass[11pt]{article}
\usepackage[utf8]{inputenc}
\usepackage[T1]{fontenc}
\usepackage{fixltx2e}
\usepackage{graphicx}
\usepackage{longtable}
\usepackage{float}
\usepackage{wrapfig}
\usepackage{rotating}
\usepackage[normalem]{ulem}
\usepackage{amsmath}
\usepackage{textcomp}
\usepackage{marvosym}
\usepackage{wasysym}
\usepackage{amssymb}
\usepackage{hyperref}
\tolerance=1000
\usepackage{listings}
\date{\today}
\title{Aces Sprint \#0 Documentation}
\hypersetup{
  pdfkeywords={},
  pdfsubject={},
  pdfcreator={Emacs 24.5.1 (Org mode 8.2.10)}}
\begin{document}

\maketitle
\tableofcontents


\section{The Team}
\label{sec-1}

\subsection{Daniel Dyla}
\label{sec-1-1}

Daniel is a strong coder with an extensive programming background
using multiple programming languages and an experienced system
administrator. His core strength is in learning new technologies
quickly and adapting them for use.

Daniel will be the sprint master for sprint \#1, as well as acting as
software engineer for the web and server components. He will also be
helping to design the protocol for use by the application.

\subsection{Gjergji Heqimi}
\label{sec-1-2}

Gjergi Heqimi is a proficient programmer in Java, C++, and C
programming languages, as well as decently skilled with Python.

Gjergji will be acting as a software engineer for the desktop client
for sprint \#1.

\subsection{Anthony Hewins}
\label{sec-1-3}

Anthony Hewins is proficient in Java, as well as reasonably familiar
with some other programming languages. He has a strong math background
and is very good at research.

Anthony will be acting as a software engineer for the server
component, in addition to helping Daniel design the chat protocol that
the application will use.

\subsection{Joshua Lindsay}
\label{sec-1-4}

Joshua Lindsay understands various programming languages including
HTML, CSS, and C. He is very organized with files and data and is a
proficient researcher.

Joshua will be acting as the user interface designer for sprint \#1

\subsection{Ryan Richardson}
\label{sec-1-5}

Ryan Richardson is competent in Java and C++.

For sprint \#1, Ryan will be acting as an interface developer for the
desktop user interface.

\subsection{Nathalie Tate}
\label{sec-1-6}

Nathalie Tate is an intermediate programmer with experience with a few
different programming languages. She is also a competent system
administrator.

For sprint \#1, Nathalie will be working on the user interface deisgn,
as well as acting as the interface developer for the web component.

\section{Rejected Projects}
\label{sec-2}

\subsection{Symmetric Key Encryption Module}
\label{sec-2-1}

Rejected because it requires too much specialized knowledge to
implement properly, as well as being too narrowly focused.

\subsection{Beer Rating and Recommendation App}
\label{sec-2-2}

Rejected because a portion of the team is not yet 21 years of age,
limiting its usefulness to some members of the team.

\subsection{Parking Monitor App}
\label{sec-2-3}

Rejected because it would have either required special hardware to be
installed in the parking lots, or a critical mass of users feeding it
data for it to be of any practical use.

\subsection{Call of Halo: FPS}
\label{sec-2-4}

Rejected because it is far beyond the scope of this class, as well as
difficulties securing rights to use the Halo and/or Call of Duty
trademarks and intellectual property.

\subsection{Class Schedule Helper}
\label{sec-2-5}

Rejected because any reasonably useful implentation would have been
far too difficult to implement over the course of a single
semester. Existing solutions to this problem are also already very good.

\subsection{Calculator}
\label{sec-2-6}

Rejected for being far too simple.

\section{Project Introduction}
\label{sec-3}

AceChat

\subsection{Description}
\label{sec-3-1}

AceChat will be a multi-user chat application with a client/server
architecture, allowing it to run on multiple platforms with minimal
duplication of work. It will have public chat rooms, as well as
private one-on-one chat between users.

\subsection{Target Devices}
\label{sec-3-2}

\subsubsection{Web}
\label{sec-3-2-1}

The web component will run using HTML, CSS, and JavaScript. It will be
a single page application, using javascript and AJAX or websockets to
communicate with the server.

\subsubsection{Desktop}
\label{sec-3-2-2}

The desktop client will be built using Java. Even though Java should
work on other platforms, we will be targeting Windows as our supported
platform.

\subsection{Required Resources and Materials}
\label{sec-3-3}

\begin{itemize}
\item Server for development and testing
\item Client code
\item Server code
\item Windows computers to test desktop application
\item Phones to test the web component on mobile
\item Internet connectivity
\item Database for data persistence
\end{itemize}

\section{Project Timeline}
\label{sec-4}

\subsection{Sprint \#1}
\label{sec-4-1}

\begin{itemize}
\item Define a basic protocol for use by the client and server
\item Set up a server for development and testing
\item Design a user interface for the client
\end{itemize}

\subsubsection{Client (Desktop and Web)}
\label{sec-4-1-1}

\begin{itemize}
\item Decide on a UI framework
\item Implement basic user interface with no functionality
\begin{itemize}
\item Depends on User interface being designed
\end{itemize}
\item Add basic chat functionality to the user interface
\begin{itemize}
\item Depends on well defined protocol
\item Depends on user interface being implemented
\end{itemize}
\end{itemize}

\subsubsection{Server}
\label{sec-4-1-2}

\begin{itemize}
\item Implement basic chat protocol for use by clients
\begin{itemize}
\item Depends on well defined protocol
\end{itemize}
\end{itemize}

\subsubsection{Deliverable by End of Sprint}
\label{sec-4-1-3}

\begin{itemize}
\item A working server implementation
\item A web client, possibly with basic chat functionality
\item A desktop client, possibly with basic chat functionality
\end{itemize}

\subsection{Sprint \#2}
\label{sec-4-2}

\begin{itemize}
\item Testing and bug squashing
\begin{itemize}
\item Depends on a partially (at least) working server and client implementation
\end{itemize}
\item Implement fully working public and private chat functionality in
desktop and client
\item Begin adding extra features such as log-in, multi-line, public
channel moderation, etc\ldots{}
\begin{itemize}
\item Depends on basic chat functionality working well
\end{itemize}
\end{itemize}

\subsection{Sprint \#3 and \#4}
\label{sec-4-3}

\begin{itemize}
\item Testing and bug squashing
\item Implement any extra features
\end{itemize}

\subsection{Sprint \#5}
\label{sec-4-4}

This sprint will be solely about testing and bug squashing, as well as
getting ready for the final presentation.

\section{Production for Sprint \#1}
\label{sec-5}

\subsection{Roles}
\label{sec-5-1}

\subsubsection{Sprint Master}
\label{sec-5-1-1}

Daniel Dyla

\subsubsection{Protocol Designer}
\label{sec-5-1-2}

\begin{itemize}
\item Daniel Dyla
\item Anthony Hewins
\end{itemize}

\subsubsection{User Interface Designer}
\label{sec-5-1-3}

\begin{itemize}
\item Nathalie Tate
\item Joshua Lindsay
\end{itemize}

\subsubsection{Software Engineer}
\label{sec-5-1-4}

\begin{itemize}
\item Server
\begin{itemize}
\item Daniel Dyla
\item Anthony Hewins
\end{itemize}
\item Desktop
\begin{itemize}
\item Gjergji Heqimi
\end{itemize}
\item Web
\begin{itemize}
\item Daniel Dyla
\end{itemize}
\end{itemize}

\subsubsection{User Interface Developer}
\label{sec-5-1-5}

\begin{itemize}
\item Desktop
\begin{itemize}
\item Ryan Richardson
\end{itemize}
\item Web
\begin{itemize}
\item Nathalie Tate
\end{itemize}
\end{itemize}

\section{Tools and Technologies}
\label{sec-6}

\subsection{Development Platform}
\label{sec-6-1}

\subsubsection{Server}
\label{sec-6-1-1}

The server will run on a linux server and be implemented using either
NodeJS and Javascript or Python and Flask. It will be a REST
interface, allowing the clients to work with it seamlessly independent
of their implementations. 

The server implementation will be developed using vim, emacs, and
sublime text as development environments in a python virtual
environment or in a NodeJS project.

The server implentation will use the Flask API framework if Python is
used. If NodeJS is used, an API framework has not yet been decided
on. It will also require libraries for communication using either
sockets or websockets as well as database libraries for data
persistence.

\subsubsection{Desktop Client}
\label{sec-6-1-2}

The desktop client will be run on a Windows computer, communicating
with the server using a REST API and/or a messaging interface over a
network socket or a websocket.

The desktop client will be implemented using the Netbeans Java IDE and
target Java 7 on the Windows desktop environment.

Our team has yet to decide on a UI framework to use for the desktop
client, but it will require libraries for communication with the
server over websockets, REST, and/or a network socket.

\subsubsection{Web Client}
\label{sec-6-1-3}

The web client will be developed in HTML, CSS, and JavaScript It will
target the Chrome web browser, but will hopefully work well with other
browsers.

The web client will use vim, emacs, and sublime text as development
environments.

The web client will require libraries for user interface, as well as
for REST API and websocket communication with the server.

\subsection{Additional Tools}
\label{sec-6-2}

\begin{itemize}
\item The GIMP Graphics Editor
\item Microsoft Paint
\item Linux
\item Microsoft Windows
\item Chrome Web Browser
\item Chrome Developer Tools
\item htop performance monitoring
\item Windows Task Manager
\end{itemize}

\section{Risk Analysis}
\label{sec-7}

We believe that we may experience problems getting the clients to work
well with the server and with each other if they are developed
entirely in isolation. In order to address this problem, we are going
to be utilizing a test driven development methodology in order to
verify that every module adheres strictly to the protocol
specification.

\section{Reference Materials}
\label{sec-8}

So far we have just used online web searches in order to find
reference materials. We believe that we will have to use
documentation for any programming languages that we use, as well as
documentation for frameworks, libraries, and toolkits that we
use. Specifically: \texttt{docs.python.org}, \texttt{devdocs.io/javascript}, and
\texttt{flask.pocoo.org/docs/0.11/}, though we are sure that there will be
others.
% Emacs 24.5.1 (Org mode 8.2.10)
\end{document}
