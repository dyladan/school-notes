% Created 2016-10-11 Tue 13:52
\documentclass[12pt]{article}
\usepackage[utf8]{inputenc}
\usepackage[T1]{fontenc}
\usepackage{fixltx2e}
\usepackage{graphicx}
\usepackage{longtable}
\usepackage{float}
\usepackage{wrapfig}
\usepackage{rotating}
\usepackage[normalem]{ulem}
\usepackage{amsmath}
\usepackage{textcomp}
\usepackage{marvosym}
\usepackage{wasysym}
\usepackage{amssymb}
\usepackage{hyperref}
\tolerance=1000
\usepackage{listings}
\usepackage{geometry,listings,amsmath,amssymb,amsthm}
\author{Daniel Dyla}
\date{\today}
\title{PHL104 Oct 11}
\hypersetup{
  pdfkeywords={},
  pdfsubject={},
  pdfcreator={Emacs 24.5.1 (Org mode 8.2.10)}}
\begin{document}

\maketitle

\section{Kant}
\label{sec-1}

\begin{itemize}
\item Autonomous will
\begin{itemize}
\item "self law" will
\end{itemize}
\item Heteronomous will
\begin{itemize}
\item "different law" will
\item "other law" will
\end{itemize}
\end{itemize}

Are you in control of your own actions?

Kant believes that actions cannot be morally praiseworthy if you are
being controlled by your heteronomous will. 

\subsection{Duty and Will}
\label{sec-1-1}

\begin{itemize}
\item An action is morally praiseworthy iff one recognizes it to be one's
duty.
\item The central question: Is the action contaminated by some
inclination?
\begin{itemize}
\item If yes, then the action proceeds from heteronomous will. It is not
determined by reason, but rather by irrational influences - things
like happiness, pleasure, and fear.
\item If no, then the action proceeds from the autonomous will. It is the
action of the self, unassisted by "other" things. It is rational.
\end{itemize}
\end{itemize}

\subsection{Hypothetical Imperative}
\label{sec-1-2}

A command that must be followed in order to obtain some desired end.

\begin{itemize}
\item If you want to be happy, then follow your dreams.
\item If you want to avoid salt, then don't eat canned soups.
\end{itemize}
% Emacs 24.5.1 (Org mode 8.2.10)
\end{document}
